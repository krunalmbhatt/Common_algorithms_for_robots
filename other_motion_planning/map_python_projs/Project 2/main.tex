\documentclass[a4 paper]{article}

% Set target color model to RGB
\usepackage[inner=1.5cm,outer=1.5cm,top=1.5cm,bottom=2.5cm]{geometry}
\usepackage{setspace}
\usepackage[rgb]{xcolor}
\usepackage{amsgen,amsmath,amstext,amsbsy,amsopn,amssymb}
\usepackage[colorlinks=true, urlcolor=blue,  linkcolor=blue, citecolor=blue]{hyperref}
\usepackage{enumitem}
\usepackage{dsfont}
\usepackage{algorithm}
\usepackage[english]{babel}
\usepackage[noend]{algpseudocode}
\usepackage{multirow}
\newcommand{\ra}[1]{\renewcommand{\arraystretch}{#1}}

\newcommand{\project}[4]{
   \pagestyle{myheadings}
   \thispagestyle{plain}
   \newpage
   \setcounter{page}{1}
   \noindent
   \begin{center}
   \framebox{
      \vbox{\vspace{2mm}
    \hbox to 6.28in { {\bf RBE550:~Motion Planning \hfill} }
       \vspace{6mm}
       \hbox to 6.28in { {\Large \hfill \textbf{#1} #2  \hfill} }
       \vspace{6mm}
       \hbox to 6.28in { {\it Student 1: #3 \hfill } }
       \hbox to 6.28in { {\it Student 2: #4 \hfill }}
      \vspace{2mm}}
   }
   \end{center}
   \vspace*{4mm}
}
\title{Motion Planning}

\begin{document}
\project{Project 2}{}{Krunal Bhatt}{Smit Shah}

$\mathbf{Theoretical Questions:}$
%%-------------------------------- QUESTION 1 -------------------------------%%
\begin{enumerate}
    \item Write two key-differences between the Bug 1 and Bug 2 algorithm. \\
    \textit{$Sol^n$:}
        \begin{description}[font=$\bullet$\scshape\bfseries]
            \item Bug 1 is an exhaustive search algorithm meaning that it looks at all the choices before choosing one, while the Bug 2 algorithm is greedy meaning that it chooses the one that comes first while searching.
            \item Bug 1 circumnavigates the obstacle and remembers the nearest point to the goal and then circles back to it, while Bug 2 has a line "M" which the bug follows and reaches the goal efficiently.
            \item Bug 2 beats Bug 1, in not all but many cases. 
        \end{description}

        %%------------------QUESTION 2 -----------------------%%
    \item A fundamental part of A* search is the use of a heuristic function to avoid exploring
unnecessary edges. To guarantee that A* will find the optimal solution, the heuristic function must be admissible. A heuristic h is admissible if 
\begin{equation*}
h(n) \leq T(n)
\end{equation*}
Where, T is the true cost from n to the goal, and n is a node in the graph. In other words, an admissible heuristic never overestimates the true cost from the current state to the goal. A stronger property is consistency. A heuristic is consistent if for all consecutive states $n$; $n_0$
    \begin{equation*}
        h(n) \leq T(n;n_0)+h(n_0)
    \end{equation*}
where T is the true cost from node n to its adjacent node $n_0$. \\
Answer the following: \\
(a) (5 points) Imagine that you are in the grid world and your agent can move up, down, left, right, and diagonally, and you are trying to reach the goal cell $f = (g_x;g_y)$. Define an admissible heuristic function $h_a$ and a non-admissible heuristic $h_b$ in terms of $x; y; g_x; g_y$. Explain why each heuristic is admissible/non-admissible.\\
(b) (10 points) Let $h_1$ and $h_2$ be consistent heuristics. Define a new heuristic $h(n)=max(h_1(n); h_2(n))$. Prove that h is consistent. \\
    \textit{$Sol^n$:}
        \begin{description}[font=$\bullet$\scshape\bfseries]
            \item For question (a), we are given a grid world and our agent can move up, down, left, right, and diagonally. We are trying to reach the goal cell \textit{f} = ($g_x$,$g_y$). In order to do that, we define an admissible heuristic $h_a$ such that: 
            \begin{equation*}
                h_a(x,y,g_x,g_y) = |x - g_x| + |y-g_y|
            \end{equation*}
            This heuristic is admissible because it never overestimates the true cost from the current state to the goal. The Manhattan distance is a lower bound on the actual cost because it only considers the horizontal and vertical movements, not diagonal movements.
            \begin{equation*}
                h_b(x,y,g_x,g_y) = (x - g_x)^2 + (y-g_y)^2
            \end{equation*}
            Here,the above function, Euclidean Distance squared will overestimate the heuristic as the square of the distance will be much more resulting into the overestimation.

            \item For question (b), we need to prove that the new heuristic h(n) = max($h_1$(n), $h_2$(n)) is consistent, we need to show that for all consecutive states n and $n_0$, the following inequality holds:
            \begin{equation*}
                h(n) \leq T(n, n_0) + h(n_0)
            \end{equation*}
            Taking max on both sides of the inequality, 
            we get,
            \begin{equation*}
                max(h_1(n), h_2(n)) \leq max(T(n, n_0) + h_1(n_0), T(n, n_0) + h_2(n_0))
            \end{equation*}
            Since the maximum of two values is always less than or equal to the sum of those values, we can simplify the inequality as follows:
            \begin{equation*}
                max(h_1(n), h_2(n)) \leq T(n, n_0) + max(h_1(n_0), h_2(n_0))
            \end{equation*}
            This shows that $h(n) = max(h_1(n), h_2(n))$ satisfies the consistency condition for all consecutive states n and $n_0$. Therefore, h(n) is consistent.
        \end{description}

%%------------------------QUESTION 3-----------------------%%
    \item Suppose you are planning for a point robot in a 2D workspace with polygonal obstacles. The start and goal locations of the robot are given. The visibility graph is defined as follows: \\
    • The start, goal, and all vertices of the polygonal obstacles compose the vertices of the graph.\\
    • An edge exists between two vertices of the graph if the straight line segment connecting the vertices does not intersect any obstacle. The boundaries of the obstacles count as edges.
    
    Answer the following:
    
    (a) (10 points) Provide an upper bound of the time it takes to construct the visibility graph in big-O notation. Give your answer in terms of n, the total number of vertices of the obstacles. Provide a short algorithm in pseudocode to explain your answer. Assume that computing the intersection of two line segments can be done in constant time.
    
    (b) (10 points) Can you use the visibility graph to plan a path from the start to the goal? If so, explain how and provide or name an algorithm that could be used. Provide an upper bound of the run-time of this algorithm in big-O notation in terms of n (the number of vertices in the visibility graph) and m (the number of edges of the visibility graph). If not, explain why.

    \textit{$Sol^n$:}
        
        \begin{description}[font=$\bullet$\scshape\bfseries]
            \item  For the answer (a), To construct the visibility graph in a 2D workspace with polygonal obstacles, we need to consider each vertex of the obstacles and check for visibility with other vertices. The upper-bound time complexity of constructing the visibility graph can be expressed in big-O notation as O($n^3$), where n is the total number of vertices of the obstacles.\
            
            Here is a short algorithm in pseudo-code to explain the construction of the visibility graph: \\
            {\centering 
\begin{minipage}{\linewidth}
\begin{algorithm}[H]
\caption{Construction of Visibility Graph}
    Initialize an Empty Graph G. \\
    Add start and goal location as vertices to G.
    \begin{algorithmic}
        \item \textbf{for} each vertex v in polygon obstacle \textbf{do} \\
            \ \  Add v as a vertex to G
            \For {each other vertex u in polygonal obstacle}
                \If {Line segment connecting v and u intersect any obstacle}
                    \State Don't add an edge and return to the start of loop
                \Else 
                    \State Add an edge connecting v and u in G
                \EndIf
                
            \Return Constructed graph G. 
            \EndFor
    \end{algorithmic}
\end{algorithm}
\end{minipage}
}

           \item For answer (b), Yes, the visibility graph can be used to plan a path from the start to the goal. The visibility graph is a graph representation of the obstacles in a given environment, where the vertices represent the points of visibility and the edges represent the unobstructed line of sight between the points.

           
           One algorithm that can be used to find a path in a visibility graph is Dijkstra's algorithm. Dijkstra's algorithm is a popular graph search algorithm that finds the shortest path between two vertices in a graph. It can be applied to the visibility graph by treating the vertices as the points of visibility and the edges as the connections between them. The algorithm is as follows: \\
           {\centering 
\begin{minipage}{\linewidth}
\begin{algorithm}[H]
\caption{Dijkstra's Algorithm}
    \begin{algorithmic}
        \item \textbf{for each} vertex v in Graph G
        \textbf{do} \\
            \ \  dist[v]:= infinite \\
            \ \ previous := undefined \\
            dist[source] := 0 \\
            Q := set of all nodes in the graph
            \While {Q is not empty}
                \State u := node in Q with smallest dist[]
                \State remove u form Q
                \For{each neighbor v of u}
                    \State alt:= dist[u] + dist$\_$between(u,v)
                        \If{alt $<$ dist[v]}
                            \State dist[v] := alt
                            \State previous[v] := u
                        \EndIf
                \EndFor
            \Return previous[]
            \EndWhile
    \end{algorithmic}
\end{algorithm}
\end{minipage}
}
The runtime of Dijkstra's algorithm in terms of the number of vertices (n) and edges (m) can be expressed as O((n + m) log n). This is because the algorithm iterates through all the vertices and edges, and the priority queue used in the algorithm has a logarithmic time complexity for the insertion and extraction of elements. 
        \end{description}

\end{enumerate}

\end{document}
